\subsection{System Model for File Synchronisation using Dropbox}\label{sec:application:cloud_file_synchronisation:system_model}
This section first provides a general overview over the Dropbox service architecture and introduces the considered use case.
Then, we propose the cloud storage model and metrics used in this analysis.
Finally, we discuss a set of scheduling mechanisms used to start the file synchronisation process.

The authors of~\cite{Drago2012} provide a first study of the \dropbox architecture, which is schematically depicted in \reffig{fig:application:cloud_file_synchronisation:system_model:dropbox_architecture} and used as a basis for the model under study in the remainder of this section.

\begin{figure}
  \centering
  \includegraphics[width=\columnwidth]{application/cloud_file_synchronization/system_model/figures/dropbox_architecture}
  \caption{\dropbox file storage and retrieval process.}
  \label{fig:application:cloud_file_synchronisation:system_model:dropbox_architecture}
\end{figure}

The \dropbox infrastructure consists of two main components:
\begin{enumerate*}
\item a storage cloud based on Amazon's Elastic Compute Cloud and Simple Storage Service, and
\item control servers directly maintained by \dropbox Inc.
\end{enumerate*}
The control servers store meta information about the current state of the files in the \dropbox folders and trigger synchronisation processes on the clients.

A file synchronisation can basically be described in five steps.
As soon as the new file is added to the \dropbox folder of the uploading client, a preprocessing step is triggered and the meta information for the file are generated, respectively updated.
This information is then synchronised with the control servers~(1) and the file itself is uploaded to the storage cloud~(2).
After the file has completely been transferred to the storage cloud, all connected clients are notified about the update~(3) and start downloading the new file~(4).

\subsubsection*{Use Case: Photo Uploading}\label{sec:application:cloud_file_synchronisation:use_case}
In this section we consider the synchronisation of images from a digital camera to a mobile \gls{UE} via a cloud storage provider.
Real world examples of this scenario are, e.g., taking photos of a live event and transferring them to a picture agency, or shooting private holiday images.

The user took a finite set of pictures with a wearable device like Google Glass or a smart camera, e.g. a Nikon Coolpix S800c or SAMSUNG CL80.
The camera is then connected via a \gls{PAN} with a mobile \gls{UE}, for example a laptop with a data card or a smartphone, to store the images on the mobile device.
The \gls{UE} uses broadband wireless Internet access technology and runs software provided by the cloud storage provider in order to synchronise the images with the cloud storage.
Finally, the scenario includes a remote client, which is connected using a wire line connection and downloads the images from the cloud.

For the evaluation we consider a specific realisation of the use case described above.
For the role of the cloud storage provider we consider \dropbox, Bluetooth is used as the technology for establishing the \gls{PAN}, and \gls{LTE} is used as the wireless broadband access technology.

In the considered scenario the interests of two stakeholders are impacted.
The first stakeholder, the end user, has two contradicting requirements on the system.
On the one hand, the images should be synchronised as fast as possible.
This requires a fast and permanent Internet connection of the \gls{UE}, which in turn is very power intensive.
On the other hand, the power drain of the mobile device should be minimised to enable a long battery life time.
The second stakeholder, the mobile network provider, wants to minimise the signalisation overhead in the network~\cite{NSN2011, Huawei2011} caused by short time connections.
Here, an optimisation problem arises to find a practical solution for all three requirements.
In order to analyse this problem, we use a simulation model of the file synchronisation process, which is described in the following.

\subsubsection*{Cloud Storage Model and Performance Metrics}\label{sec:application:cloud_file_synchronisation:system_model:model_metrics}
The proposed simulation model is schematically depicted in \reffig{fig:application:cloud_file_synchronisation:system_model:model_metrics:model} and based on the findings of~\cite{Drago2012} described in~\refsec{sec:application:background}.

\begin{figure}
\centering
\includegraphics[width=\columnwidth]{application/cloud_file_synchronization/system_model/figures/model}
\caption{Considered synchronisation process model.}
\label{fig:application:cloud_file_synchronisation:system_model:model_metrics:model}
\end{figure}

We assume that the user has taken pictures of varying file size distributed with \imageFileSize.
These pictures are transferred from the camera to the mobile device using the \gls{PAN} with a constant bandwidth~\panTransferRate.
Due to the limited bandwidth \panTransferRate of the \gls{PAN} device, the inter-arrival times of images at the \dropbox shared folder of the mobile device can be calculated by \(\interarrivaltime = \frac{\imageFileSize}{\panTransferRate}\).

As soon as the image is fully copied to the \dropbox folder, the generation of the meta data introduces a preprocessing delay, which we refer to as client preparation time~\clientpreparationtime.
To evaluate different strategies optimising the overall waiting time, power drain, and signalisation traffic we include a scheduling component.
This component implements different algorithms, which trigger sending the images currently held available in the scheduling component to the component responsible for transmission.

Next, we consider the \gls{LTE} \gls{UE} used for image upload.
Due to the specification of the \gls{LTE} standard \cite{3GPP_RRC_Spec}, the upload component can, at any point in time, either be connected to the mobile network or disconnected.
If the \gls{UE} is currently disconnected, and a new image for upload arrives, the connection process is triggered and completed after a startup delay \(\startupDelay = \SI{0.26}{\second}\).
Once the \gls{UE} is connected, arriving images are transmitted in order.
The transmission, i.e. service time, of an image depends on the size of the image currently being uploaded as well as the upload bandwidth \uploadbandwidth.
As only one image is transferred at once, waiting images are stored in a queue of infinite size.
If the \gls{UE} is idle for more than \(\idleThreshold = \SI{11.576}{\second}\), the device disconnects from the network.

After the image has been successfully uploaded to the storage servers, a server side preprocessing phase starts, before the file transfer to the downloading client starts.
This server side preprocessing again introduces an additional delay, the server preparation time~\serverpreparationtime, in the synchronisation process.
Finally, the image is downloaded by the wire line client.
Again, the duration is calculated based on the size of the image and the available download bandwidth~\downloadbandwidth.

Next, we discuss the metrics used to evaluate the performance of the scheduling algorithms under consideration.
First, we consider the mean synchronisation time \sojournTime, i.e., the time between the generation of images and the completion of the download.
This metric accounts for the desire of end users to synchronise images in a short amount of time.
Secondly, we study the relative amount of time the \gls{UE} is disconnected \relativeDisconnectedTime.
As the \gls{UE} consumes more power in the connected state, the user is generally interested in scheduling mechanisms which ensure that the device is only connected if required \cite{Ickin2012}.
This measure also enables a more general evaluation than the actually consumed power, as the concrete power drain differs significantly for each device.
Finally, we evaluate the number of transitions~\connectionCount between the connected and disconnected states.
As discussed in \refsec{chap:network}, frequent state transitions stress the network due to increased signalling.
Thus, scheduling algorithms with a small number of transitions would be favoured by network operators.

\subsubsection*{Considered Scheduling Algorithms}\label{sec:application:cloud_file_synchronisation:system_model:algorithms}
We use different scheduling strategies in our model to control the uploading of the files from the mobile client.
These mechanisms in turn affect the synchronisation time, the power drain, and the generated signalling traffic.

The most basic strategy of handing the upload is to immediately send new files, as soon as the meta data is generated.
We refer to this as the \algoimmediate strategy and will use this as base line for all comparisons in the evaluations.
The other two strategies considered are based on a temporal, respectively a size threshold.
Using the \algointerval scheduling, the client checks periodically, according to an interval \thresholdInterval, if new files have been marked for synchronisation.
If new files are present, they are synchronised to \dropbox.
Files which could not be sent within the current interval will automatically be added to the file batch for the next interval.
The last scheduling mechanisms uses a threshold \thresholdSize based on the overall \algosize of the images not yet synchronised.
If the threshold is crossed, an upload is triggered.
