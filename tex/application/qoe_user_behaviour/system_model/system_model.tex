\subsection{System Model}\label{sec:application:qoe_user_behaviour:system_model}
\subsubsection*{\(M/M/1\) Queue with \(pq\)-Policy}\label{sec:application:qoe_user_behaviour:system_model:mm1pq}
The state of the video playback is characterized by the tuple \((i, j)\), where \(i \in \{0, 1\}\) is the playback state of the client, i.e. the video is not played back if \(i\) is \(0\) and the video is played back if $i$ is $1$ and $j \geq 0$ gives the number of unplayed frames currently available at the client.
Furthermore, we give the probability of the playback being in state \((i, j)\) as \(x(i, j)\).

We obtain the following equilibrium state equations.
\begin{align*}
  \lambda x(0, 0) &= 0 & &\\
  \lambda x(0, i) &= \lambda x(0, i-1) & i&\in \left[1, q\right)\\
  \lambda x(0, q) &= \lambda x(0, q-1) + \mu x(1, q+1) & &\\
  \lambda x(0, i) &= \lambda x(0, i-1) & i&\in\left(q, p\right)\\
  (\lambda + \mu) x(1, q+1) &= \mu x(1, q+2) & &\\
  (\lambda + \mu) x(1, i) &= \lambda x(1, i-1) & i&\in\left(q+1, p\right) + \mu x(1, i+1) & &\\
  (\lambda + \mu) x(1, p) &= \lambda (x(0,p-1) + x(1, p-1)) + \mu x(1, p+1) & &\\
  (\lambda + \mu) x(1, i) &= \lambda x(1, i-1) + \mu x(1, i+1) & i&\in\left(p, +\infty\right)
\end{align*}
State probabilities are obtained using macro state equations and recursive reduction and follow analogously to~\cite{Zhang2004}:
\begin{align*}
x(0, i) &= 0 & i&\in \left[0,q\right)\\ 
x(0, i) &= \frac{1-a}{d} &i&\in \left[q,p\right)\\
x(1, i) &= \frac{a(1- a^{i-q})}{d} &i&\in \left(q,p\right]\\
x(1, i) &= \frac{a^{j-p+1}(1-a^{d})}{d} &i&\in \left(p,+\infty\right]
\end{align*}

From this we obtain the stalling ratio \stallingRatio as the probability of being in a stalling state, i.e.
\begin{equation}
\stallingRatio = \sum_{i=0}^{p-1} x(0, i) = 1-a \, .
\label{eq:application:qoe_user_behaviour:system_model:mm1pq:stalling_ratio_queuing}
\end{equation}

\subsubsection*{Mean Value Analysis of Steady State}\label{sec:application:qoe_user_behaviour:system_model:steady_state}
While the \(M/M/1-pq\) model provides results for infinite length videos, real-world videos however are of finite length.
This requires the study of additional metrics, i.e. the number of stalling events during playback \numberStallingEvents.
Thus, in this section we derive a mean value analysis of the proposed video playback model according to \reffig{fig:application:qoe_user_behaviour:system_model:steady_state:player}.

\begin{figure}
  \centering
  \includegraphics{application/qoe_user_behaviour/system_model/figures/player}
  \caption{Video buffer status evolving over time with constant video bitrate and network bandwidth for a finite video of duration \(T\) and \(Z\) frames.}
  \label{fig:application:qoe_user_behaviour:system_model:steady_state:player}
\end{figure}

Assume, that the initial download begins at \(t_0\) and new frames arrive with rate \networkBandwidth at the client.
The number of frames in the buffer exceeds \(q\) the first time at \(t_1\).
At time \(t_2\), the threshold of \(p\) is reached for the first time and playback begins.
While the download of new frames continues with rate \networkBandwidth, frames are played out with rate \playbackRate, resulting in a buffer change with rate \(\networkBandwidth - \playbackRate\).
Thus, the number of buffered frames reaches \(q\) at time \(t_3\).
This process repeats which results in an alternating chain of stalling and playback phases.

In this analysis we consider the steady state, i.e. especially neglecting the time \(t_1 - t_0\).
First, we consider the time required for the buffer to fill from \(q\) frames to \(p\) frames, i.e. obtaining \(d\) frames while no playback is occurring.
This time depicts the average duration \(\meanStallingEventDuration\) of a single stalling event.

In \reffig{fig:application:qoe_user_behaviour:system_model:steady_state:player} this is given as the time between \(t_1\) and \(t_2\), and we get
\begin{equation}
L=t_2-t_1 = \frac{p-q}{\lambda}=\frac{d}{\lambda} = \frac{d^*}{a}\, .
\label{eq:application:qoe_user_behaviour:system_model:steady_state:mean_stalling_event}
\end{equation}
The average stalling length \(\meanStallingEventDuration\) only depends on the actual buffer size \(d\) and the network bandwidth \(\networkBandwidth\).

Next, we consider the time required for the number of frames in the buffer to decrease from \(p\) 
to \(q\), i.e. the time between \(t_2\) and \(t_3\), 
\[t_3-t_2 = \frac{d}{\mu-\lambda},\]
which represents the time of uninterrupted playback between each stalling event.

Combining these two equations we get the time between two stalling event as
\[
t_3-t_1=(t_3-t_2)+(t_2-t_1)=\frac{\mu d}{(\mu-\lambda)\lambda}.
\]

The stalling ratio \stallingRatio follows as
\begin{equation}
\stallingRatio = \frac{t_2-t_1}{t_3-t_1} = 1-a,
\label{eq:application:qoe_user_behaviour:system_model:steady_state:stalling_ratio}
\end{equation}
yielding the same result as in \refeq{eq:application:qoe_user_behaviour:system_model:mm1pq:stalling_ratio_queuing}.

Finally, we can obtain the number of of stalling events normalized by video duration by analysing the busy periods of the system.
Here, the mean idle period is given by \(\meanIdle = \frac{d}{\lambda}\).

For the mean busy period \(\meanBusy\) we obtain
\[
\frac{\meanBusy}{\meanBusy + \meanIdle} = 1 - R = a \, ,
\]
which yields
\[
\meanBusy = \frac{a}{1-a}\frac{\lambda}{d} 
\]
and in turn can be used to obtain the normalized number of stalls 
\begin{equation}
N^* = \frac{1}{\meanBusy}=\frac{\mu-\lambda}{d} = \frac{1-a}{d^*}\, .
\label{eq:application:qoe_user_behaviour:system_model:steady_state:normalized_number_of_stalls}
\end{equation}

This equation can also be derived by considering \(N^*=\frac{1}{t_3-t_2}\). 
While \(N^*\) relates the stalls to the video duration, the stalling frequency \(F\) denotes the number of stalls per unit time. 
It holds \(F=\frac{1}{t_3-t_1}=a N^*\) which is also can be obtained as \(F=x(0,p-1) \lambda\) to weight the state probability of player state change \(x(0,p-1)\) with the network arrival rate \(\lambda\). 
From an end user's point of view, the metric $N^*$ is of higher importance. 

Beside the network bandwidth \(\lambda\) and the video bitrate \(\mu\), the metric \(N^*\) of stalling events depends only on the video buffer size \(d=p-q\), but not on the concrete values of \(p\) and \(q\) in the steady state.


\subsubsection*{Mean Value Analysis of Finite Videos and User Aborts}\label{sec:application:qoe_user_behaviour:system_model:finite_video}

As we will see later in \refsec{sec:application:qoe_user_behaviour:typical_user_scenarios}, the steady state analysis is sufficient to dimension the buffer. 
However, in practice, playback is finite, either because the video is of finite length \(T\), or because a user aborts playback after a number of \(T\) seconds.
This behaviour is shown in \reffig{fig:application:qoe_user_behaviour:system_model:steady_state:player}.

We do not consider the time until the initial playback, i.e. the time between \(t_0\) and \(t_2\) as stalling, since it has a much lower impact on the perceived quality than stalling \cite{Garcia2014} and it only depends on the network bandwidth \networkBandwidth and thus is not subject to optimisation.
First, we consider the case where the user plays back the complete video.
Given the network bandwidth \networkBandwidth and a video of \numberFrames Frames, the required download time for the complete video is \(\videoDownloadTime = \frac{Z}{\lambda}\).
Within \videoDownloadTime there are \(N\) phases of stalling and playback and each phase is of duration \(t_3 - t_1\).
\[
N = \left\lfloor \frac{t_Z-t_1+t_0}{t_3-t_1} \right\rfloor 
\]

Next, we consider the case where the user aborts playback of the video after \(T\) seconds of video have been watched.
Here, the number of stalling phases is given as 
\[
N = \left \lfloor{T / (t_3 - t_2)}\right \rfloor,
\]
rounding down as we do not consider the initial delay before playback as stalling.
Again, we can obtain the number of stalling events normalized by video length as
\(N^*=\frac{N}{T}\).