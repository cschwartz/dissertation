\subsection{YouTube \headershortacr{QoE} Model}\label{sec:application:qoe_user_behaviour:typical_user_scenarios:youtube_qoe}
This section introduces \gls{QoE} models for YouTube video playback.
First, we extend the \gls{QoE} mapping function introduced in~\cite{Hossfeld2013c} in order to support user preferences regarding sensitivity to stalling duration and number of stalling events.
Then, we provide a parametrised mapping function allowing for user preferences regarding initial delay.
Finally, we combine the proposed mapping functions.

\subsubsection*{Stalling \headershortacr{QoE} Model}\label{sec:application:qoe_user_behaviour:typical_user_scenarios:youtube_qoe:stalling}
The \gls{QoE} of \gls{HTTP} streaming depends mainly on the actual number of stalling events \(N\) for a video of duration \(T\) and the average length \(L\) of a single stalling event.
A \gls{QoE} model combining both key influence factors into a single equation \(f(L,N)\) is provided in~\cite{Hossfeld2013c} and found to follow the IQX hypothesis~\cite{Fiedler2010} describing an exponential relationship between the influence factors and \gls{QoE}.
In particular, the model function returns \gls{MOS} on a 5-point absolute category rating scale with 1 indicating the lowest \gls{QoE} and 5 the highest \gls{QoE}:
\begin{equation}
 f(L,N) = 3.5 e^{-(0.15L + 0.19)N}+1.50.
\label{eq:application:qoe_user_behaviour:typical_user_scenarios:youtube_qoe:stalling:original_model}
\end{equation}
Due to well known rating scale effects, the model in \refeq{eq:application:qoe_user_behaviour:typical_user_scenarios:youtube_qoe:stalling:original_model} has a lower bound of \(1.50\), as users avoid the extremities of the scale called \emph{saturation effect}, see e.g.~\cite{Moller2000}.
In contrast, if the video is not stalling, no degradation is observed and users rate the impact of stalling as 'imperceptible', i.e. a value of 5.

It has to be noted that the model function in \refeq{eq:application:qoe_user_behaviour:typical_user_scenarios:youtube_qoe:stalling:original_model} is based on subjective user studies with videos of duration up to \(T=\SI{30}{\second}\).
For other video durations, the normalised number \(N^*=\frac{N}{T}\) of stalling events has to be considered which requires to adapt the parameters \(\alpha=0.15\) and \(\beta=0.19\) in \refeq{eq:application:qoe_user_behaviour:typical_user_scenarios:youtube_qoe:stalling:original_model}, respectively.

As the goal of our investigation is the analysis of the impact of different user profiles, we parametrise the function in \refeq{eq:application:qoe_user_behaviour:typical_user_scenarios:youtube_qoe:stalling:original_model} with parameters \(\alpha\) and \(\beta\) and conduct a parameter study on their impact.
For the sake of simplicity, we normalise the QoE value to be in the range \(\left[0;1\right]\)  and use the normalised number of stalling events \(N^*\).
As a result, we arrive at \refeq{eq:application:qoe_user_behaviour:typical_user_scenarios:youtube_qoe:stalling:parameterized_model} below as a parametrised \gls{QoE} model \(Q_1\) to quantify the impact of stalling on QoE for different user profiles expressed by \(\alpha\) and \(\beta\).
Thereby, the parameter \(\alpha\) adjusts the sensitivity of the user to the stalling duration \(L\cdot N^*\), while \(\beta\) quantifies the sensitivity of the user to the actual number of stalling events, i.e. the video interruptions.
Therefore, we will also use the term \emph{duration parameter} and \emph{interruption parameter} for \(\alpha\) and \(\beta\), respectively.

The model function \(Q_1\) in \refeq{eq:application:qoe_user_behaviour:typical_user_scenarios:youtube_qoe:stalling:parameterized_model} has the same form as \refeq{eq:application:qoe_user_behaviour:typical_user_scenarios:youtube_qoe:stalling:original_model} and follows the IQX hypothesis, but allows to investigate different user profiles:

\begin{equation}
   Q_1(L,N^*) = e^{-\left( \alpha L + \beta\right) N^*}.
\label{eq:application:qoe_user_behaviour:typical_user_scenarios:youtube_qoe:stalling:parameterized_model}
\end{equation}

For example, some users may suffer stronger from interruptions which is then adjusted by a higher value of \(\beta\).
Thus, a user profile can be expressed in terms of different values of the duration parameter \(\alpha\) and the interruption parameter \(\beta\)

\subsubsection*{Initial Delay \headershortacr{QoE} Model}\label{sec:application:qoe_user_behaviour:typical_user_scenarios:initial_delay}
Another impairment on \gls{HTTP} streaming \gls{QoE} are initial delays before the video play-out can start for the first time.
The impact of initial delays \(T_0\) is modelled by the function given in \refeq{eq:application:qoe_user_behaviour:typical_user_scenarios:initial_delay:original_model}, model parameters are obtained from subjective tests \cite{Hossfeld2012c}, yielding
\begin{equation}
g(T_0)=-0.963 \cdot \mathrm{log10}(T_0 + 5.381) + 5.
\label{eq:application:qoe_user_behaviour:typical_user_scenarios:initial_delay:original_model}
\end{equation}

The results in~\cite{Hossfeld2012c} show that the impact of the initial delay is independent of the video duration which was either \SI{30}{\second} or \SI{60}{\second} in the user tests.
Further, it was observed that users have a clear preference of initial delays instead
of stalling and that service interruptions have to be avoided in any case, even at costs of increased initial delays for filling up the video buffers.

For the sake of simplicity, we normalise the function in \refeq{eq:application:qoe_user_behaviour:typical_user_scenarios:initial_delay:original_model} to obtain the \gls{QoE} model \(Q_2\) for initial delays \(T_0\), so that \(Q_2\) returns values in \(\left[0;1\right]\) and that \(Q_2(0)=1\) holds, by adding the term \(\gamma \mathrm{log10}\left(c\right)\).

The user profile is parametrised with the parameter \(\gamma\) determining the impact of initial delays on the user \gls{QoE}.
The constant \(c=5.381\) is taken from \refeq{eq:application:qoe_user_behaviour:typical_user_scenarios:initial_delay:original_model} defining the shape of the curve.
Since the logarithm is not bounded, only positive values are considered to ensure \(Q_2(T_0) \in [0;1]\):
\begin{equation*}
Q_2(T_0)= -\gamma \mathrm{log10}\left(T_0 + c\right) + \gamma \mathrm{log10}\left(c\right)+ 1.
\end{equation*}

\subsubsection*{Combined \headershortacr{QoE} Model}\label{sec:application:qoe_user_behaviour:typical_user_scenarios:youtube_qoe:combined}
For dimensioning the video buffers, we are interested in a \gls{QoE} model which considers both the impairments due to stalling and due to initial delays of the video play-out.
However, to the best of our knowledge no combined model exists so far which has been validated by proper subjective user studies.
Therefore, we suggest the following model \(Q\).
Since the impact of stalling events clearly dominates the user perception \cite{Hossfeld2012a,Hossfeld2012c}, we consider the following rationale for the combined QoE model.
A user facing an initial delay \(T_0\) experiences a \gls{QoE} value of \(Q_2(T_0)\).
If additional stalling events occur, this will lower the QoE further.
Thus, \(Q_2(T_0)\) is the upper bound of \gls{QoE}.
For \(N^*\) stalling events with an average length \(L\), the \gls{QoE} will be further decreased by \(Q_1(L,N^*)\).

An additive \gls{QoE} model for non-adaptive HTTP streaming which is referred to as buffer-related perceptual indicator is recommended in \cite{ITUT2012}. This model follows the same rationale above, start from the maximum QoE value which is \(1=Q(0,0,0)\), subtract the degradation from stalling \(1-Q_1(L,N^*)\) and \(1-Q_2(T_0)\) stemming from initial delay.

Then, we arrive at the following additive QoE model \(Q\) used in the following analysis:
\begin{eqnarray}
  Q(T_0,L,N^*) &=& 1-(1-Q_1(L,N^*)) - (1-Q_2(T_0)) \notag\\
   &=& Q_1(L,N^*) + Q_2(T_0) - 1.
\label{eq:application:qoe_user_behaviour:typical_user_scenarios:youtube_qoe:combined:qsum}
\end{eqnarray}
