\subsection{YouTube QoE Model}\label{sec:application:qoe_user_behaviour:typical_user_scenarios:youtube_qoe}
\subsubsection*{Stalling QoE Model}\label{sec:application:qoe_user_behaviour:typical_user_scenarios:youtube_qoe:stalling}
The \gls{QoE} of \gls{HTTP} streaming depends mainly on the actual number of stalling events \(N\) for a video of duration \(T\) and the average length \(L\) of a single stalling event.
A \gls{QoE} model combining both key influence factors into a single equation \(f(L,N)\) is provided in \cite{Hossfeld2013c} and found to follow the IQX hypothesis \cite{Fiedler2010} describing an exponential relationship between the influence factors and \gls{QoE}.
In particular, the model function returns \gls{MOS} on a 5-point absolute category rating scale with 1 indicating the lowest \gls{QoE} and 5 the highest \gls{QoE}. 
\begin{equation}
 f(L,N) = 3.5 e^{-(0.15L + 0.19)N}+1.50
\label{eq:application:qoe_user_behaviour:typical_user_scenarios:youtube_qoe:stalling:original_model}
\end{equation}
Due to well known rating scale effects, the model in \refeq{eq:application:qoe_user_behaviour:typical_user_scenarios:youtube_qoe:stalling:original_model} has a lower bound of \(1.50\), as users avoid the extremities of the scale called \emph{saturation effect}, see e.g. \cite{Moller2000}.
In contrast, if the video is not stalling, no degradation is observed and users rate the impact of stalling as 'imperceptible', i.e. a value of 5.

It has to be noted that the model function in \refeq{eq:application:qoe_user_behaviour:typical_user_scenarios:youtube_qoe:stalling:original_model} is based on subjective user studies with videos of duration up to \(T=\SI{30}{\second}\).
For other video durations, the normalized number \(N^*=\frac{N}{T}\) of stalling events has to be considered which requires to adapt the parameters \(\alpha=0.15\) and \(\beta=0.19\)in \refeq{eq:application:qoe_user_behaviour:typical_user_scenarios:youtube_qoe:stalling:original_model}, respectively. 

\subsubsection*{Initial Delay QoE Model}\label{sec:application:qoe_user_behaviour:typical_user_scenarios:initial_delay}
\subsubsection*{Combined QoE Model}\label{sec:application:qoe_user_behaviour:typical_user_scenarios:youtube_qoe:combined}
