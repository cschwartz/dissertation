\subsection{YouTube QoE Model}\label{sec:application:qoe_user_behaviour:typical_user_scenarios:youtube_qoe}
\subsubsection*{Stalling QoE Model}\label{sec:application:qoe_user_behaviour:typical_user_scenarios:youtube_qoe:stalling}
The \gls{QoE} of \gls{HTTP} streaming depends mainly on the actual number of stalling events \(N\) for a video of duration \(T\) and the average length \(L\) of a single stalling event.
A \gls{QoE} model combining both key influence factors into a single equation \(f(L,N)\) is provided in~\cite{Hossfeld2013c} and found to follow the IQX hypothesis~\cite{Fiedler2010} describing an exponential relationship between the influence factors and \gls{QoE}.
In particular, the model function returns \gls{MOS} on a 5-point absolute category rating scale with 1 indicating the lowest \gls{QoE} and 5 the highest \gls{QoE}. 
\begin{equation}
 f(L,N) = 3.5 e^{-(0.15L + 0.19)N}+1.50
\label{eq:application:qoe_user_behaviour:typical_user_scenarios:youtube_qoe:stalling:original_model}
\end{equation}
Due to well known rating scale effects, the model in \refeq{eq:application:qoe_user_behaviour:typical_user_scenarios:youtube_qoe:stalling:original_model} has a lower bound of \(1.50\), as users avoid the extremities of the scale called \emph{saturation effect}, see e.g.~\cite{Moller2000}.
In contrast, if the video is not stalling, no degradation is observed and users rate the impact of stalling as 'imperceptible', i.e. a value of 5.

It has to be noted that the model function in \refeq{eq:application:qoe_user_behaviour:typical_user_scenarios:youtube_qoe:stalling:original_model} is based on subjective user studies with videos of duration up to \(T=\SI{30}{\second}\).
For other video durations, the normalized number \(N^*=\frac{N}{T}\) of stalling events has to be considered which requires to adapt the parameters \(\alpha=0.15\) and \(\beta=0.19\)in \refeq{eq:application:qoe_user_behaviour:typical_user_scenarios:youtube_qoe:stalling:original_model}, respectively. 

As the goal of our investigation is the analysis of the impact of different user profiles, we parametrize the function in \refeq{eq:application:qoe_user_behaviour:typical_user_scenarios:youtube_qoe:stalling:original_model} with parameters \(\alpha\) and \(\beta\) and conduct a parameter study on their impact. 
For the sake of simplicity, we normalize the QoE value to be in the range \(\left[0;1\right]\)  and use the normalized number of stalling events $N^*$. 
As a result, we arrive at \refeq{eq:application:qoe_user_behaviour:typical_user_scenarios:youtube_qoe:stalling:parameterized_model} as parametrized \gls{QoE} model \(Q_1\) to quantify the impact of stalling on QoE for different user profiles expressed by \(\alpha\) and \(\beta\). 
Thereby, the parameter \(\alpha\) adjusts the sensitivity of the user to the stalling duration \(L\cdot N^*\), while \(\beta\) quantifies the sensitivity of the user to the actual number of stalling events, i.e. the video interruptions.
Therefore, we will also use the term \emph{duration parameter} and \emph{interruption parameter} for \(\alpha\) and \(\beta\), respectively.

\begin{equation}
   Q_1(L,N^*) = e^{-\left( \alpha L + \beta\right) N^*} 
\label{eq:application:qoe_user_behaviour:typical_user_scenarios:youtube_qoe:stalling:parameterized_model}
\end{equation}

The model function \(Q_1\) in \refeq{eq:stalling} has the same form as \refeq{eq:application:qoe_user_behaviour:typical_user_scenarios:youtube_qoe:stalling:original_model} and follows the IQX hypothesis, but allows to investigate different user profiles.
For example, some users may suffer stronger from interruptions which is then adjusted by a higher value of \(\beta\).
Thus, a user profile can be expressed in terms of different values of the duration parameter \(\alpha\) and the interruption parameter \(\beta\)

\subsubsection*{Initial Delay QoE Model}\label{sec:application:qoe_user_behaviour:typical_user_scenarios:initial_delay}
Another impairment on \gls{HTTP} streaming \gls{QoE} are initial delays before the video playout can start for the first time.
The impact of initial delays \(T_0\) is modeled by the function, model parameters are obtained from subjective tests \cite{Hossfeld2012c}. 
\begin{equation}
g(T_0)=-0.963 \: \mathrm{log10}(T_0 + 5.381) + 5
\label{eq:application:qoe_user_behaviour:typical_user_scenarios:initial_delay:original_model}
\end{equation}

The results in~\cite{Hossfeld2012c} show that the impact of the initial delay is independent of the video duration which was either \SI{30}{\second} or \SI{60}{\second} in the user tests.
Further, it was observed that users have a clear preference of initial delays instead
of stalling and that service interruptions have to be avoided in any case, even at costs of increased initial delays for filling up the video buffers. 

For the sake of simplicity, we normalize the function in \refeq{eq:application:qoe_user_behaviour:typical_user_scenarios:initial_delay:original_model} to obtain the \gls{QoE} model \(Q_2\) for initial delays \(T_0\), so that \(Q_2\) returns values in \(\left[0;1\right]\) and that \(Q2(0)=1\) holds.
The user profile is parametrized with the parameter \(\gamma\) determining the impact of initial delays on the user \gls{QoE}.
The constant \(c=5.381\) is taken from \refeq{eq:application:qoe_user_behaviour:typical_user_scenarios:initial_delay:original_model} defining the shape of the curve. 
Since the logarithm is not bounded, only positive values are considered to ensure \(Q_2(T_0) \in [0;1]\).
\begin{equation}
Q_2(T_0)= -\gamma \mathrm{log10}\left(T_0 + c\right) + \gamma \mathrm{log10}\left(c\right)+ 1 
\label{eq:application:qoe_user_behaviour:typical_user_scenarios:initial_delay:parameterized_model}
\end{equation}

\subsubsection*{Combined QoE Model}\label{sec:application:qoe_user_behaviour:typical_user_scenarios:youtube_qoe:combined}
For dimensioning the video buffers, we are interested in a \gls{QoE} model which considers both, the impairments due to stalling and due to initial delays of the video playout.
However, to the best of our knowledge no combined model exists so far which has been validated by proper subjective user studies.
Therefore, we suggest the following model \(Q\).
Since the impact of stalling events clearly dominates the user perception \cite{Hossfeld2012a,Hossfeld2012c}, we consider the following rationale for the combined QoE model.
A user facing an initial delay \(T_0\) experiences a \gls{QoE} value of \(Q_2(T_0)\).
If additional stalling events occur, this will lower the QoE further.
Thus, \(Q_2(T_0)\) is the upper bound of \gls{QoE}.
For \(N^*\) stalling events with an average length \(L\), the \gls{QoE} will be further decreased by \(Q_1(L,N^*)\).

An additive \gls{QoE} model for non-adaptive HTTP streaming which is referred to as buffer-related perceptual indicator is recommended in \cite{ITUT2012}. This model follows the same rationale above, start from the maximum QoE value which is \(1=Q(0,0,0)\), subtract the degradation \(1-Q_2(T_0)\) stemming from initial delay, and from stalling \(1-Q_1(L,N^*)\).

Then, we arrive at the following additive QoE model \(Q\) used in the following analysis.  
\begin{eqnarray}
  Q(T_0,L,N^*) &=& 1-(1-Q_1(L,N^*)) - (1-Q_2(T_0)) \notag\\
   &=& Q_1(L,N^*) + Q_2(T_0) - 1
\label{eq:application:qoe_user_behaviour:typical_user_scenarios:youtube_qoe:combined:qsum}
\end{eqnarray}
