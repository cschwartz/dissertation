\section{Lessons Learned}\label{sec:application:lessons_learned}
This chapter studied internet video streaming and cloud file synchronisation as two prominent examples of cloud backed network applications.
During the operation and use of these applications, a number of stakeholders interact, each with different outlooks on what qualifies as optimal performance of the application.
The \emph{application provider} controls the application and directly influences the performance of the application for all other stakeholders.
In case of the \emph{video streaming} scenario, we consider the application provider to be interrested in increasing the \gls{QoE} for the application user while also maintaining a low utilisation of network and compute resources, as they would incurr additional costs on his part.
In the \emph{file synchronisation} scenario the application operator is interrested in the file synchronisation occuring as soon as possible, as this has been identified as a main impact factor of the users \gls{QoE}.
The \emph{user} is interrested in perceiving a high \gls{QoE} as well as increasing the battery life of the \gls{UE}.
As in the last chapter, the \emph{network operator} is interrested to quantify the impact of application traffic on its network infrastructure.

Application operator can do
User may consist of multipe groups
Consider network operator to be passive, last chapter showed that they should not optimise for single applications

lessons learned lte video

lessons learned qoe user behaviour

lessons learned cloud file synchronisation

Conclusion