\section{Background and Related Work}\label{sec:application:background}

\subsection{Mobile Networks}

%file storage
Methods for reducing energy consumptions in \gls{LTE} Machine to Machine scenarios are considered in\scite{tirronen2012}.
The authors consider trade-offs between responsiveness and energy consumption by means of prolonging the discontinuous reception cycles in the \gls{LTE} standard.
While our work also suggest mechanisms to decrease energy consumption, no modifications in the \gls{LTE} standard are required.

%lte journal
The authors of \citep{siekkinen2013} perform a measurement of power consumption and \gls{ran} signalling during playback of a \youtube video.
They employ a proxy server in order to ensure that traffic is sent in bursts, thus decreasing power consumption at the cost of additional signalling traffic.

\subsection{Video Streaming Mechanisms}\label{sec:application:background:video_streaming_mechanisms}

%lte journal
In order to match the demand of video transmission over the Internet, multiple solutions exist \citep{begen2011}.
The most basic approach, \download, obtains the complete video at once, playing back any available content as required.
Due to the nature of \live video transmissions it is only possible to send the currently available content.
Furthermore, introducing delay into the live-stream should be avoided as it reduces the timeliness of the video.
There exist different approaches for \streaming video content to a user.
In server based solutions, the streaming server controls the transmission of content.
One example of such a server based approach is the \gls{rtsp} which was widely discussed as a standardized solution for mobile video streaming \citep{elsen2001}.

In the more recent past, client based approaches were discussed.
Here the client controls the download and playback of content.
The authors of \citep{oyman2012} study the \gls{qoe} of \gls{has} approaches, if the content is consumed via \gls{lte} networks.
They highlight the differences to existing server-side approaches and suggest the study of cross-layer optimization approaches in order to improve the \gls{qoe}.
One approach to deliver \gls{has} is \dash, which enables video streaming over \http \citep{sodagar2011}.
Considering both the video content as well as the available resources by using a proxy has been suggested to improve the users \gls{qoe} \citep{essaili2013}.
In \citep{xin2012} the authors suggest the use of a caching strategy, downloading video content according to a user viewing history and network conditions.

\subsection{Application Quality of Experience}\label{sec:application:background:application_qoe}

%file storage
A general study of the \gls{QoE} influence factors of file storage servies is undertaken\scite{qian2011}.
Here, the authors provide a model allowing for the evaluation of cloud service providers according to a variety of metrics, including bandwidth, latency, and response time.
In\scite{amrehm2013} the authors study the main impact factors on \gls{QoE} of Dropbox users.
They find that the main impact factor is the waiting time for file synchronization.