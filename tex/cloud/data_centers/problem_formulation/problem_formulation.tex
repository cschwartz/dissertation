\subsection{Problem Formulation}\label{sec:cloud:data_centers:problem_formulation}

A widely used data center architecture is the three-tier architecture shown in \reffig{fig:cloud:data_centers:problem_formulation:3-tier_datacenter}.
The upper two layers of the architecture are responsible for distributing the traffic and consist of layer 3 switches where each switch has a backup switch.
In this paper, we focus on the edge layer and here on a single \gls{POD}.
A \gls{POD} consists of a number of servers connected over top of rack switches to an aggregation switch.

\begin{figure}
  \centering
  \includegraphics{cloud/data_centers/problem_formulation/figures/architecture}
  \caption{Three-tier data center architecture}
  \label{fig:cloud:data_centers:problem_formulation:3-tier_datacenter}
\end{figure}


We assume, that new jobs entering the system arrive with exponentially distributed inter-arrival time.
When a job in form of a packet arrives at the \gls{POD}, it is forwarded to an idle server.
If no idle server is available, the job is queued.
Once a server finishes processing its current job, it picks another one from the queue.

Our goal is now to evaluate how much power is consumed in a data center and how much can be saved when servers, not processing any job, are switched off.
Therefore, we developed two different data center models.
The first model, the \emph{default data center}, consists of two-state servers only which are either \emph{busy} or \emph{idle}, as shown in \reffig{fig:cloud:data_centers:problem_formulation:servers:idle_busy}) 
For the second model, a more \emph{energy-efficient data center}, a subset of the servers may additionally be switched on and \emph{off} on demand, shown in \reffig{fig:cloud:data_centers:problem_formulation:idle_busy_off} as recommended in~\cite{EPA2007}.

\begin{figure}
	\begin{subfigure}[b]{\textwidth}
	\centering
	\includegraphics{cloud/data_centers/problem_formulation/figures/idle_busy}
	\caption{2-state server model}\label{fig:cloud:data_centers:problem_formulation:servers:idle_busy}
	\end{subfigure} 
	\begin{subfigure}[b]{\textwidth}
	\centering
	\includegraphics{cloud/data_centers/problem_formulation/figures/idle_busy_off}
	\caption{3-state model of a reserved server}\label{fig:cloud:data_centers:problem_formulation:idle_busy_off}
	\end{subfigure}

	\caption{Power state transition on a per server level}\label{fig:cloud:data_centers:problem_formulation:servers}
\end{figure}

\subsection{Default Data Center}\label{sec:cloud:data_centers:problem_formulation:default_data_center}
For the default data center model, each of the \(n\) servers is either on and processing a job or on and idle as depicted in \reffig{fig:cloud:data_centers:problem_formulation:servers:idle_busy}.
If a busy server finishes processing a job and the queue is empty, the server becomes idle. Once a new job is assigned to a yet idle server, the server becomes busy.
According to our measurements of a server with an Intel twelve core processor \SI{2.67}{\giga\hertz} and \SI{32}{\giga\byte} RAM, a server currently processing a job consumes \(e_{\text{busy}} = \SI{240}{\watt}\)
An idle server still consumes \(e_{\text{idle}} = \SI{170}{\watt}\).

\subsection{Energy-Efficient Data Center}\label{sec:cloud:data_centers:problem_formulation:energy_efficient_data_center}
For the second model, we differentiate between two types of servers.
\(n\) base-line servers which are always on and \(m\) reserved servers to be enabled on demand.
If they are enabled, their power consumption is similar to that of the default data center model.
If they are disabled, each server consumes \(e_\text{off} = \SI{0}{\watt}\).
The \(n\) servers which are always enabled consume the same power as in the default data center model.
If the system queue has a length exceeding \(\theta_2\) where \(\theta_2 \in (0, m)\) holds, the \(m\) reserved servers are enabled and stay enabled until the total number of jobs in the system drops to \(\theta_1\) for \(\theta_1 \in (0, n)\).
The transition between power levels for each of the reserved servers is depicted in \reffig{fig:cloud:data_centers:problem_formulation:idle_busy_off}.

The energy-efficient data center operation model with the parameters \(\theta_1\) and \(\theta_2\) is depicted in \reffig{fig:cloud:data_centers:problem_formulation:model} and described in detail in the next section.

\begin{figure}
  \centering
  \includegraphics{cloud/data_centers/problem_formulation/figures/model}
  \caption{System model for an energy-efficient operation}
  \label{fig:cloud:data_centers:problem_formulation:model}
\end{figure}