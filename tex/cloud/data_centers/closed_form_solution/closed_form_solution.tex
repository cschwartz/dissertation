\subsection{Analysis of Proposed Data Centre Models}\label{sec:cloud:data_centers:closed_form_solution}
Using the macro state equations discussed in \refsec{sec:cloud:data_centers:modeling}, there are multiple ways to obtain the state probabilities.
This section examines the different approaches.

\subsubsection*{System of Linear Equations}
First, it is possible to directly solve the system of linear equations implied by the micro or macro states.
Solvers for linear equation systems scale cubic in the dimension of the matrix, which in this case is bounded by the system size.
Especially for a large numbers of server, this prevents an exhaustive search of the parameter space.

\subsubsection*{Closed-form Solutions}
Thus, we obtain closed form solutions for the state probabilities.
These equations can be derived by recursively applying the macro state equations.

All equations feature a factor \(x(0, 0)\) which in turn can be calculated using the normalisation property given in \refeq{eq:cloud:data_centers:modeling:normative}.
Due to the length of the individual formulas, the following shorthand is introduced:
For each state probability \(x(i, j)\) depending on the factor \(x(0,0)\) we define \(\bar{x}(i, j) = x(i, j) \cdot x(0, 0)^{-1}\), i.e. we cancel the factor.

For \(0 < i <\theta_1\), we get
\begin{align*}
x(i, 0) &= x(0, 0) \cdot \frac{a^i}{i!}.
\end{align*}
As a further shorthand for substitution, we define
\begin{align*}
s_i&=\sum_{k=0}^{i}a^k(n-k-1)!.
\end{align*}
Using this definition, we get the state probability for \(\theta_1\) jobs in the system with activated reserved servers as
\begin{align*}
x(\theta_1, 1) &= x(0, 0)\cdot \frac{a^{n+\theta_2+1}}{\left(1+\frac{a}{\theta_1}\right)}
\cdot\frac{\left(\theta_1-1\right)!\left(\frac{(1-a^{\theta_2})\theta_1}{1-a}+a^{\theta_2}s_{n-\theta_1}\right)}{n^{\theta_2}n!\left(n-\theta_1+1\right)!}.
\end{align*}
For \(\theta_1\leq i\leq n\), we get
\begin{align*}
x(i, 0) = x(0, 0) &\cdot \left(\frac{\bar{x}(n, 0) a^{i-\theta_1+1}}{(i-\theta_1+1)!} - \frac{\bar{x}(\theta_1,1) \theta_1 s_{i-\theta_1}}{i!}\right).
\end{align*}
And for \(n<i\leq n+\theta_2\), we get
\begin{align*}
x(i, 0) = x(0, 0) &\cdot \left(\frac{\bar{x}(n,0) a^{i-\theta_1+1}}{n^{i-n}(n-\theta_1+1)!}\right.\\
&\left.-\bar{x}(\theta_1,1)\left(\frac{\theta_1 s_{n-\theta_1} a^{i-n}}{n!} + \frac{\theta_1(1-a^{i-n})}{1-a}\right)\right).
\end{align*}

Thus, we have all probabilities for system states where only the baseline servers are active. For the reserved servers, we obtain state probabilities for \(\theta_1<i\leq n+\theta_2+1\) as
\begin{align*}
x(i, 1) = x(0, 0) \cdot \left(\bar{x}(\theta_1, 1)\frac{a^{i-\theta_1}\theta_1!}{i!} + \bar{x}(n+\theta_2,0)\sum_{k=1}^{i-\theta_1}\frac{a^k(i-k)!}{i!}\right).
\end{align*}

For \(n+\theta_2+1 < i \leq n + m\), we get
\begin{align*}
x(i, 1) = x(0, 0)\cdot \bar{x}(n+\theta_2+1,1) \cdot\frac{a^{i-(n+\theta_2+1)}(n+\theta_2+1)!}{i!},
\end{align*}
and finally for \(i>n+m\)
\begin{align*}
x(i, 1) = x(0, 0)\cdot \bar{x}(n+m,1) \left(\frac{a}{n+m}\right)^{i-(n+m)}.
\end{align*}
As discussed earlier, the probability of an empty system is given by the normalisation condition:
\begin{align*}
x(0, 0)=\left(1 +\sum_{k=1}^{n+\theta_2}\bar{x}(k, 0) + \sum_{k=\theta_1}^{\infty}\bar{x}(k,1)\right)^{-1}.
\end{align*}

While these closed form solutions allow for the derivation of analytical properties of the model, performing a numerical analysis of a given parameter space is difficult due to numerical instability of the equations.

\subsubsection*{Recursive Algorithm}
To avoid these problems, we introduce a recursive algorithm to calculate the state probabilities based on the macro state equations.
To this end, we first define \(x(0,0)\) as a constant \(K_0\), and then iteratively compute the state probabilities.
For an earlier application of this concept, see \cite{Trangia1997}.
First, we calculate \(x(i,0)\) for \(0<i<\theta_1\) as a factor of \(x(0,0)\) using \refeq{eq:cloud:data_centers:modeling:energy_efficient:S1_1}.
To obtain the probability for \(x(\theta_1,0)\), not only \(x(\theta_1 -1, 0)\), but \(x(\theta_1, 1)\) is required, which we have not obtained yet.
As this is the case with all \(x(i,0)\) for \(\theta_1 \leq i \leq n + \theta_2\) we implicitly introduce a second constant \(K_1\) for \(x(\theta_1, 1)\) and calculate all \(x(i,0)\) for \(\theta_1 \leq i \leq n + \theta_2\) as a linear combination of \(x(\theta_1 - 1,0)\) and \(K_1\) as follows:
\begin{equation}
x(i, 0) = x(\theta_1 - 1, 0) u_i + K_1 v_i.\label{eq:cloud:data_centers:modeling:closed_form_solution:linear_combination}
\end{equation}
For \(i = \theta_1\), \refeq{eq:cloud:data_centers:modeling:energy_efficient:S1_2} requires \(u_{\theta_1} = \frac{a}{\theta_1}\) and \(v_{\theta_1} = 1\).
Continuing this pattern by successively applying \refeq{eq:cloud:data_centers:modeling:energy_efficient:S1_2} for \(\theta_1 < i \leq n\), we get
\begin{align*}
u_i &= \frac{a}{i}u_{i-1},\\
v_i &= \frac{a}{i}v_{i-1} + \frac{\theta_1}{i}.
\end{align*}
We use \refeq{eq:cloud:data_centers:modeling:energy_efficient:S1_3} to continue for \(n < i \leq n + \theta_2\), and get
\begin{align*}
u_i &= \frac{a}{n}u_{i-1},\\
v_i &= \frac{a}{n}v_{i-1} + \frac{\theta_1}{n}.
\end{align*}
Thus, we arrive at a probability for \(x(n + \theta_2,0)\) depending on \(x(\theta_1 -1,0)\), which we have obtained, and \(K_1 = x(\theta_1,1)\) which we still need to acquire:
\begin{equation*}
x(n+\theta_2,0) = x(\theta_1 -1,0) u_{n+\theta_2} - K_1 v_{n+\theta_2}.
\end{equation*}
We apply \refeq{eq:cloud:data_centers:modeling:energy_efficient:S3} and solve for \(K_1\) and obtain
\begin{equation*}
K_1 =  \frac{\frac{a}{\theta_1} u_{n+\theta_2}}{\frac{a}{\theta_1} v_{n + \theta_2} + \theta_1} x(\theta_1 - 1, 0),
\end{equation*}
which allows to calculate the probabilities of \(x(i,0)\) for \(\theta_1 \leq i \leq n + \theta_2\) using \refeq{eq:cloud:data_centers:modeling:closed_form_solution:linear_combination}.

We can now obtain the probabilities for states in which the reserved servers have been activated, beginning with \((\theta_1 + 1,1)\) we apply \refeq{eq:cloud:data_centers:modeling:energy_efficient:S2_1} for all \(\theta_1 < i \leq n + \theta_2 + 1\) and get
\begin{equation*}
x(i,1) = \frac{a}{i}(x(i-1,1)+ x(n+\theta_2,0))
\end{equation*}
which we can calculate directly as all probabilities are known in relation to \(K_0\).
We continue applying \refeq{eq:cloud:data_centers:modeling:energy_efficient:S2_2} and obtain
\begin{equation}
x(i,1) = \frac{a}{i}(x(i-1,1) + x(n+\theta_2,0))\label{eq:cloud:data_centers:modeling:energy_efficient:probability_greater_nm}
\end{equation}
for \(n + \theta_2 + 1 < i \leq n + m\).

Finally, we need to calculate the probability that the system is in states \((i,1)\) for \(i > n + m\) where we need to obtain
\begin{equation*}
x(i>n+m,1) = \sum_{i = n + m + 1}^{+ \infty} x(i,1).
\end{equation*}
Due to the recursive definition of \refeq{eq:cloud:data_centers:modeling:energy_efficient:S2_3}, we can write
\begin{equation*}
x(i,1) = \rho x(i-1,1) = x(n + m,1)\rho^{i-(n+m)}
\end{equation*}
for \(\rho = \frac{a}{n + m}\) and \(i > n + m\).

Applying this redefinition to \refeq{eq:cloud:data_centers:modeling:energy_efficient:probability_greater_nm} we get
\begin{align*}
x(i>n+m,1) &= \sum_{i = n + m + 1}^{+ \infty} x(i, 1)\\
&= x(n+m,1)\sum_{i=1}^{+\infty} \rho^i.\nonumber
\end{align*}
After applying the properties of the geometric series and basic transformations we get
\begin{equation*}
x(i>n+m,1) =x(n + m,1) \frac{2-\rho}{1-\rho}.
\end{equation*}
Now that all probabilities are known in relation to \(K_0\), we apply \refeq{eq:cloud:data_centers:modeling:normative} to obtain the inverse of \(K_1\) and norm our values to obtain the real probabilities.

This approach allows for a fast and numerically stable calculation of the state probabilities and can be used to compute the required performance metrics for the complete parameter space.
