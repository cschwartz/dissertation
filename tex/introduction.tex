\chapter{Introduction}\label{chap:introduction}

Todays' Internet is no longer only controlled by a signular stakeholder, e.g. a standard body or a telecommunications company.
Rather, the interests of a multitude of stakeholders, e.g. application developers, hardware vendors, cloud operators, and network operators, clash during the development and operation of applications in the mobile internet. 
Each of these stakeholders considers different \glspl{KPI} to be important and attempts to optimise scenarios in its favor. 

One example of such a scenario are \emph{Signalling Storms}~\cite{Huawei2011}, with one of the largest occurring in Japan in 2012\footnote{\url{https://www.techinasia.com/docomo-outage}, Accessed: November, 21st 2015} due to the release and fast uptake of a free instant messaging application.
The connection establishment algorithms of \glspl{UE} triggered by the applications traffic resulted in a loss of service for 2.5 million users over 4 hours.
While the network operator suffers the largest impact of this signalling behaviour, it is in no good position to intervene.
The stakeholders who could prevent or at least reduce such behaviour, i.e. application developers or hardware vendors, have no direct benefit from modifying their products in such a way.  

The goal of this monograph is to provide an overview over the complex structures of stakeholder relationships in todays Internet applications.
To this end, we study different scenarios where such stakeholder interests clash and suggest methods where stakeholder tradeoffs can be optimised for all participants or detail on why such an optimisation is not possible or attempts at it might lead to adverse effects.

In the remainder of this chapter we first discuss the stakeholders considered in this work in \refsec{sec:introduction:considered_stakeholders}.
Then, in \refsec{sec:introduction:scientific_contribution} we provide an overview over the scientific contributions of this thesis.
Finally, \refsec{sec:introduction:outline} provides an outline of this monograph.

\section{Considered Stakeholders}\label{sec:introduction:considered_stakeholders}

In this section we introduce the stakeholders considered in the remainder of this monograph.
\reffig{fig:introduction:stakeholders} shows the considered stakeholders and their interactions.

\begin{figure}
\centering
\includegraphics{figures/stakeholders}
\caption{Stakeholder interactions considered in this monograph. Solid lines show stakeholder interactions, dotted lines show \emph{is-a} relationships.}\label{fig:introduction:stakeholders}
\end{figure}

First, we consider the \emph{network operator}.
The network operator owns, manages and operates a mobile network.
By manipulating network configuration, the operator can influence the connection state of \gls{UE}, resulting in changes of power drain, i.e. battery life, of the \gls{UE} and reduced signalling load in the components of the mobile network.

The \emph{application provider} develops and deploys applications and is interested in increasing the \gls{QoE} for the user, in order to attract a large user base.
Additional considered \glspl{KPI} are cost, for example incurred due to use of compute or network resources in \gls{IaaS} or \gls{PaaS} scenarios.
Design and configuration of applications has impact on traffic patterns which, result in signalling traffic in the mobile network and also influence the connection state of the \gls{UE}.

%TODO: Video Provider, Storage Provider

\glspl{UE} are developed and sold by \emph{hardware vendors}.
While they theoretically implement standards proposed by the \gls{3GPP} in order to establish connectivity with mobile networks, in reality vendors are free to deviate from standard, in order increase own \glspl{KPI}.
One example of such a deviation from a standard are proprietary fast dormancy mechanisms~\cite{GSM2010} implemented by some hardware vendors.
These algorithms reduce power drain be disconnecting the \gls{UE} earlier from the network, in order to reduce power drain and increase user \gls{QoE}.
However, this has the consequence of increased signalling in the mobile network and can result in increased page load times, i.e. decreased \gls{QoE} from the users point of view.

\emph{End users} employ \glspl{UE} to execute applications in the network of network operator. 
They are usually interested in increasing their \gls{QoE}, i.e. by increasing the battery life of their \gls{UE} or increasing satisfaction during video playback over the network.

\emph{Cloud operators} provides services, i.e. compute, storage, or network resources, to cloud users according to specified \glspl{SLA} for monetary compensation. 
They attempt to reduce the costs to operate infrastructure, e.g. by reducing power drain, in order to increase revenue while still satisfying the \gls{SLA}.

Resources provided by cloud operators are purchased by \emph{cloud users}.
They attempt to provide the best possible service to their own customers while reducing the number of required resources provided by the cloud operator, in order to reduce cost of operation. In this monograph we consider two exemplary cloud users, which will be discribed in the following:

A \emph{\gls{NFV} operator} uses virtualised resources obtained from a cloud operator to provide virtualised network services to other stakeholders.
In the example considered in this thesis, the \gls{NFV} operator uses cloud compute resources in order to provide a \gls{GGSN} to a mobile network operator.
The \glspl{KPI} of the \gls{NFV} operator are satisfying the \gls{SLA} with the mobile network operator and reducing the use of compute resources of the cloud operator.

The second considered cloud user is the \emph{crowdsourcing platform operator} who uses cloud resources in order to provide a crowd sourcing platform.
The crowdsourcing platform operator in turn has to consider the requirements of its two main stakeholders:

The \emph{crowdsourcing employer} requires a set of microtasks to be completed in a small amount of time, in order to use the generated results in future business processes.

\emph{Crowdsourcing workers} complete microtasks for a set amount of money.
They are interested in completing as many tasks as possible and reduce their idle time, in order to increase their income.

The interactions of this stakeholders result in complex interactions, which are studied in this thesis.
The next section introduces the considered interactions and provides an overview over the scientific contributions provided by this monograph.

\section{Scientific Contribution}\label{sec:introduction:scientific_contribution}
This monograph studies the interactions between different stakeholders in three, partially overlapping, scenarios in order to paint a broad picture of todays interlocking network and application ecosystem.

\begin{figure}
\centering
\includegraphics{figures/publications}
\caption{Contribution of this work as a classification of the research studies conducted by the author}\label{fig:introduction:publications}
\end{figure}

In \reffig{fig:introduction:publications} we classify the areas of research as well as scientific method used in relation to the chapter of this monograph.
Annotations of are used to highlight scientific publication who's content contributes to the respective chapters.

The first contribution of this monograph is a study of the impact of mobile application traffic on mobile communication networks, especially considering the current network configuration.
To this end, we introduce an algorithm to infer power drain and generated signalling messages , relevant \glspl{KPI} for the mobile network operator and the end user, for a given application traffic trace and network configurations.
We then extend on this algorithm and present a theoretical model to derive said metrics from arbitrary \gls{IID} traffic arrival distributions.
Using these methods we study the impact of different traffic types and study the potential of network parameter optimisation as a mean to reduce signalling traffic.

As a second contribution, we provide models for two popular applications, i.e. Video Streaming and Cloud File Synchronisation, thus enabling the study of the impact of different mechanisms implemented in said applications.
For the Video Streaming application we use a simulation in order to compare \glspl{KPI} for all stakeholders in this scenario.
We show that the Streaming mechanism allows the most flexible configuration and can provide a Pareto optimal results for all pairs of metrics.
However, further study shows that in fact no pareto optimal value exists which satisfies the \glspl{KPI} of all participating stakeholders.
Furthermore, we provide a queueing model for video streaming and use it to derive parameterisable \gls{QoE} models for different user groups.
Finally, for the Cloud File Synchronisation scenario we obtain measurements of the performance of the Dropbox service and use them as input for a simulation model.
Using this simulation model we compare different upload scheduling algorithms and find that both the Size based algorithm as well as the Time based algorithm can be used to specify a tradeoff between the different considered \glspl{KPI}.

As a third contribution, we discuss the impact of resource dimensioning and management schemes in cloud environments.
To this end, we first study the performance of a power conservation mechanism for cloud environments using a queueing model.
We derive guidelines for selecting Pareto optimal results regarding both waiting time before a job can begin processing and power drain of the cloud.
Then, we discuss a mechanism to reduce cost for cloud users by disabling compute instances while still allowing configurable \glspl{SLA} and evaluate this mechanism using a queueing simulation.
Finally, we present a model to dimension worker numbers in a human cloud scenario which can be used to ensure satisfaction the key stakeholders of a crowdsourcing platform operator.
We evaluate this model using \gls{DES} and data obtained from a major crowdsourcing platform operator. 

\section{Outline of Thesis}\label{sec:introduction:outline}

In \refchap{chap:network} we study the impact of mobile network configuration settings on participating stakeholders, i.e. mobile network operators, hardware vendors, and users.
%To this end, we first provide an overview over \gls{3G} mobile communication technology and discuss related work.
We present an algorithm to infer power drain and signalling messages caused by a given application traffic measurement.
Then, we perform application traffic measurements and discuss general traffic characteristics before applying the introduced algorithm to the measurements.
Based on the inferred power drain and signalling messages, we discuss the impact of network configuration settings and proprietary deviations from mobile communication standards.
We generalise our results by introducing a theoretical model for power drain and signalling messages for arbitrary \gls{IID} traffic distributions.
%TODO das muss neu
We apply this model to a set of analytical traffic distributions and derive general results for the considered  metrics. 

\refchap{chap:application} focusses on the impact of applications design and algorithm choice by the application developers on the other stakeholders.
We study two prominent applications in todays Internet: Video Streaming and Cloud File Sychronisation.
%video transmission mechanisms hatte einen neuen namen?
For Video Streaming, we study the impact of different video transmission mechanisms and parameter configurations on energy consumption of the \gls{UE}, signalling in the mobile network and resource consumption at the application developer using a \gls{DES}.
%auch das hatte einen anderen namen
Furthermore, we study provide a queueing model for video streaming algorithms and derive a parameterised \gls{QoE} model.
We use this model in order to provide algorithm configuration guidelines in order to optimise \gls{QoE} for different user profiles.
Next, we discuss different scheduling algorithms for Cloud File Synchronisation services.
Using data obtained from large scale, testbed based~\cite{Chun2003} measurements, we implement a simulation model in order to investigate the impact of the synchronization scheduling mechanism and configuration on time required to complete synchronisation, signalling in the mobile network and power drain.  

In \refchap{chap:cloud} we study resource allocation strategies in the cloud and evaluate in which matter decisions of the cloud platform operators impact the other stakeholders.
First, we consider a energy saving scheme where a cloud operator scales the number of available servers according to the available load.
We introduce a queueing model and perform a performance evaluation in order to study optimal parameter settings.
Then, we consider the role of a cloud user renting virtual machines in the cloud in order to provide a service to users on the example of a virtualised network operator.
This cloud user is interrested in ensuring the \gls{SLA} offered to its customers is satisfied while reducing costs by disabling servers currently not in use.
We analyse traffic characteristics and use them as input for a simulation model of a virtualised \gls{GGSN}.
Then, we evaluate the impact of different virtual server setups and scaling configurations.
Finally, we consider resource allocation in human clouds.
Based on data obtained from a commercial crowdsourcing provider, we extract characteristic distributions and apply it as input two both an analytic queueing model as well as a simulation model.
We evaluate both the impact of chosen analysis method as well as fit of the distributions obtained from the measurements.
Finally, we provide system dimensioning guidelines for the crowdsourcing provider.

Finally, \refchap{chap:conclusion} provides a summary of the major contributions of this work and provides an outlook on future potential research directions.