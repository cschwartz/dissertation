\section{Lessons Learned}\label{sec:network:lessons_learned}
In this chapter we studied the impact of smartphone application traffic on mobile communication networks.
We considered three stakeholders of interest interacting in the mobile network.
The mobile network operator is interested in preventing so called signalling storms, where network components performance is degraded due to high signalling load.
The hardware vendor is interested in satisfying customers by providing a long battery lifetime, i.e. reducing power drain.
The application developer is interested in increasing \gls{QoE} for the applications user.
Each of the stakeholders can influence the mobile network, by manipulating parameters under its control.
The network operator can manipulate \gls{RRC} timers, increasing the time a smartphone stays connected to the network if no data is sent or received and thus decreasing the number of connections being established or severed. 
The hardware vendor can implement proprietary \gls{RRC} protocol extensions, skipping power intensive connection states in order to reduce power drain.
The application developer can shorten update interval, in order to provide more up to date events and increase \gls{QoE}.
However, each of the parameters under control of the individual stakeholders influence the \glspl{KPI} of the other stakeholders.

This chapter provides a two-pronged approach to analysing the impact of changes by individual stakeholders on the overall network.

First, we provided an algorithm to derive \gls{RRC} state transitions from traffic measurements from already deployed or prototyped applications.
While proprietary mechanisms exist to directly measure \gls{RRC} state transitions, due to the high price they are usually out of reach for application developers, preventing them from evaluating the impact of their applications on the network.
Based on this algorithm we analyse four popular smartphone applications, and find that while it is possible to find a viable trade-off between signalling load and power drain for single applications, no such trade-off exists if multiple applications operating in the network at the same time are considered.
Furthermore, we show that network timer optimization, a practice where network operators manipulate \gls{RRC} timers in order to reduce signalling load, insentiences users to enable proprietary fast dormancy algorithms, resulting in a net increase of signalling load.
We also show, that while longer \gls{RRC} timers may have an adverse effect on power drain due to the smartphone being longer connected to the network, it results in an increase of Web \gls{QoE}, as this results in web pages being able to be loaded faster if the smartphone is already connected to the network.

Second, we propose an analytical model to derive the \glspl{KPI} from analytical or empirical traffic distributions, in order to evaluate the impact of applications that do not yet exist or classes of applications defined by a common traffic characteristics.
% results from 2.2

Concluding from this chapter, we see that
%network timer optimization funktioniert nicht, wir brauchen interfaces, kooperation, etc