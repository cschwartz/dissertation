\subsection{Inferring State Transitions and Deriving Metrics}\label{sec:network:network_traces:performance_evaluation}
A \gls{UE}’s firmware triggers \gls{RRC} state transitions based on application traffic.
While solutions exist to capture RRC state transitions on specific hardware~\cite{zayas2010} they are not available for all modern smartphone platforms.
Other options to measure the required information include using costly hardware and use specific \glspl{UE}, usually not available to researchers and application developers.
This prevents the developers from evaluating the effect of their applications on the overall health
of the network.
Consequently, they can not take measures to prevent the harmful behaviour of their applications.
However, it is possible to infer the \gls{RRC} state transitions for a given packet trace if the network configuration is known.

First, we describe the setup used to capture network packet traces for arbitrary apps.
Then, we give an algorithm to infer the \gls{RRC} state transitions for a given packet trace.
Based on these state transitions, we can calculate the number of signalling messages generated
by the packet trace. 
Finally, we use the information on when which \gls{RRC} state was entered to calculate the power drain of the \gls{UE}’s radio interface.

\subsubsection*{Measurement Procedure and Setup}\label{sec:network:network_traces:performance_evaluation:measurement}
To investigate the behaviour of the application under study, we capture traffic during a typical use of the application on a \emph{Samsung Galaxy SII} smartphone.
The smartphone runs the Android operating system and is connected to the \gls{3G} network of a major German network operator.
To obtain the network packet traces we use the \texttt{tcpdump} application.
This application requires \emph{root} privileges which are obtained by rooting the device and installing the custom \emph{cyanogenmod} ROM \footnote{http://www.cyanogenmod.org}.
Once \texttt{tcpdump} is installed and running, we start the application under study and capture packet traces while the application is running.
Then, the \emph{android debugging bridge} is used to copy the traces to a workstation.
The traces contain \gls{IP} packets embedded in Linux Cooked Captures.
We require the \gls{IP} packets, thus we extracted the \gls{IP} packets which are used during the analysis to follow.

\subsubsection*{Inferring Network State}\label{sec:network:network_traces:performance_evaluation:inferring_network_state}
In this section we study the influence of the application traffic on \gls{RRC} state transitions and signalling messages.
Since \gls{RRC} state transitions can not be captured using commonly available tools, we introduce an algorithm to infer \gls{RRC} state transitions from \gls{IP} packet traces.
Using this algorithm we analyse the \gls{RRC} state transition frequency and signalling message load for the Two State Model and Three State Model.

Traffic below the network layer can not be measured without specific equipment which interfaces with the proprietary firmware of the \gls{UE} and is often out of reach for developers interested in assessing the impact of their applications on the network.
Based on the Two State and Three State models introduced in \refsec{sec:network:background:umts_rrc}, we process \texttt{tcpdump} captures of the application traffic.
However, it should be noted that this method is not restricted to a specific network model, but can be extended to any other network model as well.
Using these captures, we extract the timestamps when \gls{IP} packets are sent or received.
Furthermore, we require the timer values of the transition from \gls{RRC_DCH} state to \gls{RRC_FACH} state, \gls{TDCH}, and the timer for the transition between \gls{RRC_FACH} and \gls{RRC_idle} states, \gls{TFACH}.
Based on these informations \refalg{alg:network:network_traces:performance_evaluation:inferring_network_state:inference_algorithm} infers the timestamps of state transitions according to the \gls{3GPP} specification \cite{3GPP_RRC_Spec} for the Three State Model.
This algorithm can be simplified to also work for the Two State Model. 
Alternatively, a method to post process the results of the algorithm to obtain results for the Two State Model is given at the end of this section.
The algorithm first computes the inter-arrival times of all packets.
Then, each timestamp is considered.
If the \gls{UE} is currently in \gls{RRC_idle} state, a state transition to \gls{RRC_DCH} occurs at the moment the packet is sent or received.
If the inter-arrival time exceeds the \gls{TDCH} timer the \gls{UE} transitions to \gls{RRC_FACH} \gls{TDCH} seconds after the packet was sent or received.
Similarly, if the inter-arrival time exceeds both the \gls{TDCH} and \gls{TFACH} timers a state transition to \gls{RRC_idle} occurs \gls{TDCH} seconds after the state transition to \gls{RRC_FACH}.

\begin{algorithm}
  \begin{algorithmic}
    \Require{Packet arrival timestamps \emph{ts}\\
    \gls{RRC_DCH} to \gls{RRC_FACH} timer \gls{TDCH}\\
    \gls{RRC_FACH} to \gls{RRC_idle} timer \gls{TFACH}}
    \Ensure{Times of state transition \emph{state\_time}\\
    New states after state transitions \emph{state}}
    \State \texttt{interarrival(i)} $\leftarrow$ \emph{ts}(i+1) - \emph{ts}(i)
    \State \texttt{index} $\leftarrow 0$
    \ForAll{ts(i)}
      \If{\texttt{state(index)} = \gls{RRC_idle}}
        \State \texttt{index} $\leftarrow$ \texttt{index} + 1
        \State \texttt{state(index)} $\leftarrow$ \gls{RRC_DCH}
        \State \texttt{state\_time(index)} $\leftarrow$ ts(i)
      \EndIf
      \If{\texttt{interarrival}(i-1) $> \gls{TDCH}$}
        \State \texttt{index} $\leftarrow$ \texttt{index} + 1
        \State \texttt{state(index)} $\leftarrow$ \gls{RRC_FACH}
        \State \texttt{state\_time(index)} $\leftarrow$ ts(i) $+ \gls{TDCH}$
      \EndIf
      \If{\texttt{interarrival}(i-1) $> \gls{TDCH} + \gls{TFACH}$}
        \State \texttt{index} $\leftarrow$ \texttt{index} + 1
        \State \texttt{state(index)} $\leftarrow$ \gls{RRC_idle}
        \State \texttt{state\_time(index)} $\leftarrow$ ts(i) $+ \gls{TDCH} + \gls{TFACH}$
      \EndIf
    \EndFor
  \end{algorithmic}
  \caption{Inferring \headershortacr{RRC} state transitions based on \headershortacr{IP} timestamps}
  \label{alg:network:network_traces:performance_evaluation:inferring_network_state:inference_algorithm}
\end{algorithm}

Decreasing power drain of their devices is always a goal of \gls{UE} vendors.
A straightforward way to achieve this, if only the wellbeing of the \gls{UE} is considered, is to transition from \gls{RRC_DCH} state to \gls{RRC_idle} as soon as no additional data is ready for sending.
While this transition is not directly available in the 3GPP specification for the \gls{RRC} protocol \cite{3GPP_RRC_Spec}, a \gls{UE} may reset the connection, effectively transitioning from any state to \gls{RRC_idle}.
This behaviour can be modeled using the Two State Model introduced in \refsec{sec:network:background:umts_rrc}.

State transitions for the Two State Model can be calculated using a similar algorithm.
Alternatively, the behaviour of the Two State Model can be emulated using \refalg{alg:network:network_traces:performance_evaluation:inferring_network_state:inference_algorithm} if \gls{TFACH} is set to \SI{0}{\second} and all state transitions to \gls{RRC_FACH} are removed in a post processing step.

\subsubsection*{Calculating Signalling Frequency and Power Consumption}\label{sec:network:network_traces:calculating_metrics}

\begin{table}
\centering
  \caption{Number of signalling messages per \headershortacr{RRC} state transition perceived at the \headershortacr{RNC} (Taken From \cite{3GPP_RRC_Spec})}
  \label{tab:network:network_traces:calculating_metrics:signalling_messages}
\begin{tabular}{cccc}
	\toprule
    from/to & \gls{RRC_idle} & \gls{RRC_FACH} & \gls{RRC_DCH}\\
    \midrule
    \gls{RRC_idle} & -- & 28 & 32\\
    \gls{RRC_FACH} & 22 & -- & 6\\
    \gls{RRC_DCH} & 25 & 5 & --\\
    \bottomrule    
	\end{tabular}
\end{table}

In reality, the number of state transitions is not the metric of most importance if network signalling is to be evaluated.
Each state transition results in a number of \gls{RRC} messages between the \gls{UE} and different network components.
For this study we consider the number of messages observed at the \gls{RNC}, which can be found in \cite{3GPP_RRC_Spec} and is summarized in \reftab{tab:network:network_traces:calculating_metrics:signalling_messages}.
It can be seen that transitions from or to the \gls{RRC_idle} state are especially expensive in terms of number of messages sent or received.
This is due to the fact that upon entering or leaving the \gls{RRC_idle} state, authentication has to be performed. 
Note that for the Two State Model only transitions from or to the \gls{RRC_idle} state occur.
This results in the fact that for the same network packet trace the number of signalling messages occurring in the Two State Model is generally higher than in the Three State Model.
To obtain the total number of signalling messages, we weight the number of state transitions with the number of messages sent per state transitions.
Then, we average the number of state transitions over the measurement duration to obtain a metric for the signalling load at the \gls{RNC}, i.e. the \gls{SF}.
The inference algorithm does not differentiate between state changes caused by upstream or downstream traffic.
State changes caused by downstream traffic usually generate some additional signalling messages, as paging is involved.
The inference algorithm can easily be enhanced to support this behaviour.
However, the results discussed in the next section would only change quantitatively.
Furthermore, the algorithm can be adapted to new networking models or other numbers of signalling messages sent per state transition.

\begin{table}
  \centering
  \caption{Power consumption of the \headershortacr{UE} radio interface depending on current \headershortacr{RRC} state (taken from \cite{Qian2011a})}
  \label{tab:network:network_traces:calculating_metrics:power_consumption}  
  \begin{tabular}{cc}
  	\toprule
    \gls{RRC} State & Power Consumption\\
    \midrule
    \gls{RRC_idle} & \SI{0}{\milli\watt}\\
    \gls{RRC_FACH} & \SI{650}{\milli\watt}\\
    \gls{RRC_DCH} & \SI{800}{\milli\watt}\\
    \bottomrule
  \end{tabular}
\end{table}

From a users point of view, the signalling message frequency is of little importantance.
The user is interested in a low power drain as this increases the battery life of the device.
To calculate the battery life, we use the time when state transitions occurred, and the information about the state the transition was to, to calculate the relative amount of time that was spent in each state.
Given the relative time spent in each state, we use \reftab{tab:network:network_traces:calculating_metrics:power_consumption}, taken from \cite{Qian2011a}, to compute the \gls{PD} of the radio interface during the measurement phase.
We focus on the power drain of the radio interface, as it is possible to measure the aggregated power drain using out of the box instrumentation techniques provided by the hardware vendor.